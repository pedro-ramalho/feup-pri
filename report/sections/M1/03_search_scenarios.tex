\subsection{Search Scenarios} \label{search-scenarios}
Prospective search tasks and information needs are fundamental concepts in this type of project.

Information needs refer to the underlying reasons or motivations for conducting a search. They are the goals that users are trying to achieve when using an information retrieval system. These can vary widely, and might include tasks like finding specific facts, making decision, solving problems, or simply satisfying curiosity.

As an example, consider the following information needs:
\begin{itemize}
    \item[\textbf{IN1}] \textit{A user wants to research the toxic effects of fungi of the \textit{Amanita} genus.}
    \item[\textbf{IN2}] \textit{A user is looking for scientific articles or abstracts related to diseases associated with fungi species.}
\end{itemize}

To satisfy \textbf{IN1}, the user might start by entering a query like '\textit{toxic fungi}' onto our system. After submitting their query, the user will receive a list of search results, where they may find relevant information. There may be a scientific article which satisfies this exact need, which talks about toxic and poisonous fungi (for example, \textit{Pholiotina rugosa}). The abstract of said article may be within the reach of our system.
